% Tiivistelmät tehdään viimeiseksi. 
%
% Tiivistelmä kirjoitetaan käytetyllä kielellä (JOKO suomi TAI ruotsi)
% ja HALUTESSASI myös samansisältöisenä englanniksi.
%
% Avainsanojen lista pitää merkitä main.tex-tiedoston kohtaan \KEYWORDS.

\begin{fiabstract}
  Tämä on TIK.kand-kurssin \LaTeX{}-pohja, jota voi vapaasti
  käyttää. Koko zip-paketin voi ladata kurssin Noppa-sivulta
  \url{https://noppa.tkk.fi/noppa/kurssi/TIK.kand/materiaali/}.
  Mukana on esimerkkejä \LaTeX{}:n käytöstä.

  Teksti on peräisin TIK.kand-kurssin historiallisesta
  \LaTeX{}-pohjasta, jota kurssin koordinaattori Jukka Parviainen
  päivitti tammikuussa 2011.  Lisäksi suomen kielen lehtori Sanni
  Heintzmann kirjoitti rakenteellisia vinkkejä.

  Tiivistelmä on muusta työstä täysin irrallinen teksti, joka
  kirjoitetaan tiivistelmälomakkeelle vasta, kun koko työ on
  valmis. Se on suppea ja itsenäinen teksti, joka kuvaa olennaisen
  opinnäytteen sisällöstä. Tavoitteena selvittää työn merkitys
  lukijalle ja antaa yleiskuva työstä. Tiivistelmä markkinoi työtäsi
  potentiaalisille lukijoille, siksi tutkimusongelman ja tärkeimmät
  tulokset kannattaa kertoa selkeästi ja napakasti. Tiivistelmä
  kirjoitetaan hieman yleistajuisemmin kuin itse työ, koska teksti
  palvelee tiedonvälitystarkoituksessa laajaa yleisöä.

%Tiivistelmätekstiä tähän (\languagename). Huomaa, että tiivistelmä tehdään %vasta kun koko työ on muuten kirjoitettu.
\end{fiabstract}

%\begin{svabstract}
%  Ett abstrakt hit 
%%(\languagename)
%\end{svabstract}

\begin{enabstract}
  Here goes the abstract 
(\languagename)
\end{enabstract}
