% Tiivistelmät tehdään viimeiseksi. 
%
% Tiivistelmä kirjoitetaan käytetyllä kielellä (JOKO suomi TAI ruotsi)
% ja HALUTESSASI myös samansisältöisenä englanniksi.
%
% Avainsanojen lista pitää merkitä main.tex-tiedoston kohtaan \KEYWORDS.

\begin{fiabstract}
Käytännön harjoittelulla on merkittävä rooli ohjelmoinnin opetuksessa ja
kurssit ovat usein suuria. Automaattisella arvostelulla on tässä suuri
oppimista tehostava vaikutus. Toisaalta tuntemattoman ohjelmakoodin
suorittaminen on aina riski. Tämä työ tutkii kahta TKK:lla kehitettyä
kotitehtävätarkistinta turvallisuusnäkökulmasta ja selvittää, voiko staattista
analyysiä lisäämällä parantaa näiden turvallisuutta. Lisäksi tutkitaan
opetettavan ohjelmointikielen vaikutusta tarkistimeen liittyviin riskeihin.

TKK:lla kehitetyt tarkistimet EXPACA ja {\scmrobo} ovat lähestymistavoiltaan
hyvin erilaisia. Edellinen suorittaa käännettyä ohjelmaa virtuaalikonsolissa
ja jälkimmäinen suorittaa Scheme-proseduuria metasirkulaarisessa tulkissa.
{\scmrobo} on onnistuneesti rajannut kielen riskialttiit ominaisuudet pois
eikä siinä siksi ole ilmeisiä turvallisuusriskejä. EXPACA suorittaa
täysimittaisia ohjelmia, jotka on toteutettu virhealttiilla kielillä. Tämän
takia staattisen analyysin lisäämisestä tarkistusprosessiin voisi olla hyötyä.

Staattiset analysaattorit antavat yleisesti ottaen niin paljon vääriä
hälytyksiä, ettei niiden käyttö automaattisessa arvostelussa ole mielekästä.
Pscan ja UNO ovat analysaattoreita, jotka havaitsevat yhteensä neljän tyyppisiä
ongelmia. Ne tuottavat keskimääräistä vähemmän vääriä hälytyksiä ja soveltuvat
mahdollisesti EXPACAn laajentamiseen.

Työssä havaitaan myös, että opetettavalla ohjelmointikielellä ja staattisen
analyysin opetuksella voi olla suuri merkitys opiskelijoiden tietämykselle
turvallisesta ohjelmointitavasta.
\end{fiabstract}

%\begin{svabstract}
%  Ett abstrakt hit 
%%(\languagename)
%\end{svabstract}

%\begin{enabstract}
%  Here goes the abstract. FIXME
%(\languagename)
%\end{enabstract}
