
% --------------------------------------------------------------------

\section{Johdanto}
\label{sec:johd}

Tämä kandidaatintyö tutkii tuntemattoman lähdekoodin staattista
analyysiä ja siitä käännetyn ohjelman ajonaikaista tarkkailua tavoitteena
arvioida, onko ohjelman suorittaminen turvallista. Tutkimuksen tavoitteena on
kartoittaa ohjelmointitehtävien automaattisten tarkistimien toteutukseen
liittyviä turvallisuushaasteita ja yrittää löytää ratkaisuja kirjallisuudesta.

\subsection{Tavoite}

Työn tavoitteena on löytää menetelmiä, jolla voidaan vahvistaa
ohjelmointiharjoitusten arvosteluun käytettäviä työkalujen turvallisuutta, jotta ne selviäisivät
tahattomasti ja tahallisesti tehdyistä hyökkäyksistä eheinä ja
toimintakykyisinä. Järjestelmän olennaisia vaatimuksia ovat nopea palaute,
väärien hälytysten minimointi ja käytettävyys. (Joku lähde tai parempi
tavoiteasettelu tähän?) Vahvennusmenetelmät eivät saa merkittävästi haitata
järjestelmän toimintaa.

Nopealla palautteella tarkoitetaan yksittäisen harjoituspalautuksen
tarkistamiseen kuluvaa läpimenoaikaa. Läpimenoajan tulisi olla sellainen, että
opiskelija pystyy tarkastuttamaan harjoituksensa järjestelmässä ilman työnkulun
katkaisevaa keskeytystä.

Väärä hälytys tarkoittaa sellaista virhehavaintoa, joka johtaa opiskelijan
antaman harmittoman syötteen hylkäämiseen.

\subsection{Rajaus}

Ohjelmointiharjoituksien ohjelmakoodia voidaan arvioida ennen ja jälkeen
käännöksen. Nykyaikaiset haittaohjelmat pyrkivät vaikeuttamaan koodin
automaattista analysointia salaamalla ja pakkaamalla jo käännettyä ohjelmaa.
Tällaiset menetelmät on rajattu tämän työn ulkopuolelle, koska lähdekoodin
kirjoittaja ei pääse vaikuttamaan ohjelmakoodiin sen käännön aikana tai sen
jälkeen.

Ohjelmakoodin staattiseen analyysiin liittyy useita vaikeasti ratkeavia
ongelmia. Esimerkiksi ei ole olemassa tunnettua yleisesti pätevää
tapaa osoittaa ohjelmallisesti, päättyykö ohjelman suoritus koskaan.
Automaattisesti tarkistettavat ohjelmointiharjoitukset ovat kuitenkin yleensä
ratkaisuja hyvin rajattuun ja tunnettuun ohjelmointiongelmaan. Ohjelman pitkäksi
venyvä suoritusaika on riittävä peruste ratkaisun hylkäämiselle.

\citet{heffleymeunier2004} tutkivat ohjelmien haavoittuvuuksia etsiviä työkaluja
ja havaitsivat, että monet haavoittuvuuksia tutkivat työkalut tuottavat niin
paljon vääriä varoituksia, etteivät ne ole käytännöllisiä. Työkalujen joukosta
kuitenkin erottui yksi hyödyllinen työkalu, Pscan, joka onnistui antamaan
luotettavia tuloksia rajatulla alueella. 

Monet ohjelmakoodin staattiseen analyysiin liittyvät ongelmat ovat ihmiselle
helppoja. Osa ohjelmakoodia tarkastelevista työkaluista etsii epäilyttäviä
kohtia ja antaa ne ihmisen tutkittavaksi. \citep{taft2008} Puoliautomaattiset
menetelmät eivät kuitenkaan kuulu tämän työn tutkimusalueeseen, koska ne
estävät automaattisesta tarkistamisesta saatavan nopean palautteen edun ja
ovat työläitä erityisesti suurilla kursseilla.

Koodin ajaminen virtualisoiduissa tai emuloiduissa ympäristöissä on vaativa
ongelma-alue \citep{kesti2010}. Tämä työ ei ota kantaa virtualisointiin
liittyviin ongelmiin, mutta on silti suositeltavaa ajaa kotitehtävätarkistinta
rajoitetussa ympäristössä.

% --------------------------------------------------------------------

\section{Aineisto ja menetelmät}
\label{sec:aineisto}

Ohjelmointiharjoituksen tarkistimessa voidaan varautua tietoturvaongelmiin
kahdella tasolla: huolehtimalla tarkistimen turvallisuudesta ja tutkimalla
syötteenä tulevaa ohjelmakoodia kriittisesti.

(Pitäisikö olla joku referenssitoteutus? Webcatin tutkimuksessa on keskitytty
oppimisvaikutusten tutkimukseen.)

\subsection{Staattinen analyysi}


\subsubsection{Pscan}

(Oiskohan hyvä kerätä tietoa työkaluista, joista on apua itse tarkistimen
vahvistamisessa?)

\subsection{Sallitun toteutuksen rajaaminen}

\subsubsection{Kielen rajaaminen}

(Voiko käytettyä ohjelmointikieltä rajoittaa siten, että selkeästi haitalliset
komennot eivät ole saatavilla?)

\subsubsection{Koodin ominaisuuksien rajaaminen}

(Voiko mielivaltaisilla rajoilla, kuten koodin pituudella, rajapinnoilla,
testikattavuudella, kompleksisuudella tai vastaavilla vaikuttaa niihin
vihamielisiin kikkoihin, jotka mahtuvat mukaan?)

\subsection{Ajonaikainen analyysi}

\subsubsection{Debuggeri}

(Saavutetaanko koodin ajamisella debuggerissa etua, jos jotain jäi huomaamatta?
Mitä haasteita debuggaamiseen liittyy?)

% --------------------------------------------------------------------

\section{Tulokset}
\label{sec:tulos}

% --------------------------------------------------------------------

\section{Johtopäätökset}
\label{sec:paketointi}

Ohjelmointitehtävien automaattinen arvostelussa on käytettävyyssyistä tehtävä kompromisseja.
Järjestelmät varautuvat yksinkertaisimpiin ongelmiin erittäin hyvin ja yleensä ne kestävät
opiskelijan vahingosta johtuvia hyökkäyksiä hyvin. 

\pagebreak
\input{suttu}
\pagebreak
