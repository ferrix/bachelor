% -------------- Symbolit ja lyhenteet --------------
%
% Suomen kielen lehtorin suositus: vasta kun noin 10-20 symbolia
% tai lyhennettä, niin käytä vasta sitten.
%
% Tämä voi puuttuakin. Toisaalta jos käytät paljon akronyymejä,
% niin ne kannattaa esitellä ensimmäisen kerran niitä käytettäessä.
% Muissa tapauksissa lukija voi helposti tarkistaa sen tästä
% luettelosta. Esim. "Automaattinen tietojenkäsittely (ATK) mahdollistaa..."
% "... ATK on ..."
%
\section*{Käytetyt lyhenteet}
%\section{Abbreviations and Acronyms}

\begin{tabular}{p{0.2\textwidth}p{0.6\textwidth}}
2k/4k/8k mode & COFDM operation modes \\
3GPP  & 3rd Generation Partnership Project; Kolmannen sukupolven 
matkapuhelupalvelu \\ 
ESP & Encapsulating Security Payload; Yksi IPsec-tietoturvaprotokolla \\ 
\end{tabular}

Jos tarvitset useampisivuista taulukkoa, kannattanee käyttää 
esim. \verb!supertabular*!-ympäristöä, josta on kommentoitu esimerkki.


